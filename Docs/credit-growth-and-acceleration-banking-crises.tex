\documentclass[11pt]{article} % <--- sin twocolumn

\usepackage[utf8]{inputenc}
\usepackage[T1]{fontenc}
\usepackage[spanish,english]{babel}
\usepackage[table]{xcolor}
\usepackage{adjustbox}
\usepackage{amsfonts}
\usepackage{amsmath}
\usepackage{amssymb}
\usepackage{array}
\usepackage{booktabs}
\usepackage{caption}
\usepackage{enumitem}
\usepackage{fancyhdr}
\usepackage{float}
\usepackage{geometry}
\usepackage{graphicx}
\usepackage{hyperref}
\usepackage{lipsum}
\usepackage{mathtools}
\usepackage{multirow}
\usepackage[numbers]{natbib}
\usepackage{ragged2e}
\usepackage{setspace}
\usepackage{siunitx}
\usepackage{subcaption}
\usepackage{tabularx}
\usepackage{tcolorbox}
\usepackage{threeparttable}
\usepackage{tikz}
\usepackage{titlesec}
\usepackage{url}

% --- Configuración de formato general
\setstretch{1.05} 
\titlespacing*{\section}{0pt}{*0.25}{*0.25} 
\titlespacing*{\subsection}{0pt}{*0.25}{*0.25} 
\setlength{\parskip}{0.25cm} 
\setlength{\parindent}{0pt} 
\captionsetup[figure]{labelfont=bf} 
\captionsetup[table]{labelfont=bf}  
\captionsetup{labelformat=simple, labelsep=period} 
\renewcommand{\tablename}{Tabla} 

\geometry{
	a4paper,
	left=25mm,
	right=25mm,
	top=25mm,
	bottom=25mm
}

% --- Encabezado y pie
\pagestyle{fancy}
\fancyhf{} 
\fancyhead[L]{Facultad de Ciencias Económicas (UBA) - Finanzas Internacionales} 
\fancyhead[R]{Informe Final} 
\fancyfoot[C]{\thepage} 
\renewcommand{\headrulewidth}{0.4pt} 
\renewcommand{\footrulewidth}{0pt} 

% --- Documento
\begin{document}
	
	\begin{center}
		\Large Universidad de Buenos Aires \\[0.2cm]
		\large Facultad de Ciencias Económicas \\[0.2cm]
		Escuela de Estudios de Posgrado \\[0.4cm]
		\Large \textbf{CRECIMIENTO Y ACELERACIÓN DEL CRÉDITO BANCARIO COMO PREDICTORES DE CRISIS BANCARIAS} \\[0.4cm]
		\textbf{Informe Final} \\[0.4cm]
		\normalsize \textbf{DOCENTE:} Prof. Nicolás Bertholet \\[0.6cm]
		\Large \textbf{Asignatura:} Finanzas Internacionales \\[1cm]
		\normalsize \textbf{ALUMNO:} Julián Alberto Delgadillo Marín \\[0.2cm]
		\textbf{Posgrado:} Maestría en Economía Aplicada \\[0.4cm]
		10 de octubre de 2025
		\vspace*{1cm}
	\end{center}
	
	% ===========================================================
	% RESUMEN EN ESPAÑOL
	% ===========================================================
	\begin{center}
		{\Large\textbf{Resumen}}
	\end{center}
	
	\noindent
	Este informe evalúa si el crecimiento del crédito bancario y, en particular, su \textit{aceleración} (segunda diferencia) anticipan la ocurrencia de crisis bancarias sistémicas. Se construye un panel país-año combinando (i) crédito al sector privado no financiero (BIS), (ii) variables macroeconómicas de control (WDI/IMF/GMD) y (iii) años de crisis bancarias (Laeven \& Valencia). La estrategia empírica estima modelos binarios (probit/logit) con efectos fijos por país y efectos por año, y errores agrupados por país; la validez predictiva se contrasta mediante la curva ROC (AUROC) y se interpreta la magnitud económica con efectos marginales. Los resultados muestran que tanto el crecimiento como la aceleración del crédito incrementan la probabilidad de crisis en el corto plazo, siendo la aceleración un predictor más informativo que el crecimiento simple. Las conclusiones son robustas a especificaciones alternativas y sugieren el valor de incorporar métricas de aceleración del crédito en tableros macroprudenciales de alerta temprana.
	
	\vspace{0.3cm}
	
	\noindent\textbf{Palabras clave:} crédito bancario; aceleración del crédito; crisis bancarias; alerta temprana; AUROC; efectos marginales; probit; logit; panel de países; política macroprudencial.
	
	\vspace{1cm}
	
	% ===========================================================
	% ABSTRACT IN ENGLISH
	% ===========================================================
	\selectlanguage{english}
	\begin{center}
		{\Large\textbf{Abstract}}
	\end{center}
	
	\noindent
	This report assesses whether the growth of bank credit and, in particular, its \textit{acceleration} (second difference) can predict the occurrence of systemic banking crises. A country–year panel is constructed by combining (i) credit to the non-financial private sector (BIS), (ii) macroeconomic control variables (WDI/IMF/GMD), and (iii) systemic banking crisis years (Laeven \& Valencia). The empirical strategy estimates binary models (probit/logit) with country and year fixed effects, clustering errors at the country level. Predictive validity is evaluated through the ROC curve (AUROC), and economic significance is interpreted using marginal effects. The results show that both credit growth and acceleration increase the probability of crises in the short term, with acceleration being a more informative predictor. The findings are robust across alternative specifications and support the inclusion of credit acceleration metrics in macroprudential early warning dashboards.
	
	\vspace{0.3cm}
	
	\noindent\textbf{Keywords:} bank credit; credit acceleration; banking crises; early warning; AUROC; marginal effects; probit; logit; panel data; macroprudential policy.
	
	\selectlanguage{spanish}
	
	\section{Introducción}
	\subsection{Motivación y contribución}
	\subsection{Preguntas de investigación e hipótesis}
	\subsection{Principales resultados (vista previa)}
	\subsection{Estructura del artículo}
	
	\section{Marco conceptual y literatura relacionada}
	\subsection{Ciclos financieros, apalancamiento y crisis}
	\subsection{Crédito como indicador temprano}
	\subsection{Medición: crecimiento vs. aceleración del crédito}
	\subsection{Aportes frente a trabajos previos}
	
	\section{Datos}
	\subsection{Fuentes y cobertura}
	\subsubsection{Crédito al sector privado (BIS)}
	\subsubsection{Crisis bancarias (Laeven \& Valencia)}
	\subsubsection{Controles macro (IMF/WDI/GlobalMacroData)}
	\subsection{Construcción de variables}
	\subsubsection{Nivel, crecimiento y aceleración del crédito}
	\subsubsection{Variable dependiente (dummy de crisis)}
	\subsubsection{Controles y transformaciones}
	\subsection{Muestreo y tratamiento de faltantes}
	\subsection{Estadísticos descriptivos}
	\subsubsection{Tablas de resumen}
	\subsubsection{Correlaciones y mapa de cobertura}
	
	\section{Estrategia empírica}
	\subsection{Modelo base}
	\subsubsection{Especificación (probit/logit con efectos fijos)}
	\subsubsection{Identificación y supuestos}
	\subsection{Validación y evaluación predictiva}
	\subsubsection{Métrica AUROC y curva ROC}
	\subsubsection{Clasificación y umbrales}
	\subsubsection{Efectos marginales e interpretación económica}
	\subsection{Diseño de robustez}
	\subsubsection{Rezagos, ventanas temporales y recortes}
	\subsubsection{Conjuntos alternativos de controles}
	\subsubsection{Panel balanceado vs. desbalanceado}
	
	\section{Resultados}
	\subsection{Parámetros principales}
	\subsubsection{Crecimiento del crédito}
	\subsubsection{Aceleración del crédito}
	\subsection{Importancia económica (efectos marginales)}
	\subsection{Desempeño predictivo (AUROC y clasificación)}
	\subsection{Heterogeneidad}
	\subsubsection{Avanzadas vs. emergentes}
	\subsubsection{Periodos pre/post crisis globales}
	
	\section{Robustez y pruebas adicionales}
	\subsection{Especificaciones alternativas}
	\subsection{Mediciones alternativas del crédito}
	\subsection{Placebos y sensibilidad de umbrales}
	\subsection{Colinealidad y diagnósticos}
	
	\section{Discusión e implicaciones de política}
	\subsection{Métricas de aceleración en supervisión}
	\subsection{Buffers contracíclicos y umbrales prudenciales}
	\subsection{Limitaciones y alcances}
	
	\section{Conclusiones}
	\subsection{Hallazgos clave}
	\subsection{Contribución y agenda futura}
		


	
\end{document}
